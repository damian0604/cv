\documentclass[11pt,a4paper,sans]{moderncv} 
\moderncvstyle{banking}                             % style options are 'casual' (default), 'classic', 'oldstyle' and 'banking'
\moderncvcolor{blue}                               % color options 'blue' (default), 'orange', 'green', 'red', 'purple', 'grey' and 'black'
\usepackage[utf8]{inputenc}

\usepackage{eurosym}

\usepackage[scale=0.75]{geometry}

% make subsecion headings larger (anders vallen ze weg in de literatuurlist
\renewcommand*{\subsectionfont}{\Large\mdseries\upshape}


%next two lines for page numbers
\usepackage{lastpage}
\rfoot{\addressfont\itshape\textcolor{gray}{\thepage/\pageref{LastPage}}}
%\rfoot{\addressfont\itshape\textcolor{gray}{Page \thepage\ of \pageref{LastPage}}}

%\setlength{\hintscolumnwidth}{3cm}                % if you want to change the width of the column with the dates
%\setlength{\makecvtitlenamewidth}{10cm}           % for the 'classic' style, if you want to force the width allocated to your name and avoid line breaks. be careful though, the length is normally calculated to avoid any overlap with your personal info; use this at your own typographical risks...




% personal data
\name{Damian}{Trilling}
\title{CV}
\address{Rietzangerweg 9}{1111VG Diemen}{Netherlands}
\phone[mobile]{+31~6 48 13 35 76}  
\email{d.c.trilling@uva.nl}
\homepage{www.damiantrilling.net} 
\social[twitter]{damian0604}
\social[github]{damian0604}      
%\extrainfo{additional information}       
%\photo[64pt][0.4pt]{bestandsnaam}



% zie https://tex.stackexchange.com/questions/161897/moderncv-newline-within-a-cventry-in-the-banking-style
% maakt multi-line cventry mogelijk
% of 13cm de juiste keuze is moet nog blijken
\renewcommand*{\cventry}[7][.25em]{
	\begin{tabular*}{\textwidth}{p{13cm}@{\extracolsep{\fill}}r}%
		{\bfseries #4} & {\bfseries #5} \\%
		{\itshape #3\ifthenelse{\equal{#6}{}}{}{, #6}} & {\itshape #2}\\%
	\end{tabular*}%
	\ifx&#7&%
	\else{\\\vbox{\small#7}}\fi%
        \par\addvspace{#1}}

        
% to show numerical labels in the bibliography (default is to show no labels); only useful if you make citations in your resume
%\makeatletter
%\renewcommand*{\bibliographyitemlabel}{\@biblabel{\arabic{enumiv}}}
%\makeatother
%\renewcommand*{\bibliographyitemlabel}{[\arabic{enumiv}]}% CONSIDER REPLACING THE ABOVE BY THIS




\usepackage{etaremune}
% bibliography with mutiple entries
\usepackage{multibib}
\newcites{art,thes,conf,textbook,chapter,review,misc}{{Articles},{Theses},{Conference Presentations and Posters},{Textbooks},{Book chapters},{Manuscripts under review},{Other publications}}




\begin{document}

% generate all citations now already to avoid messing with the order in case I cite somehwere in the manuscript
\nocitechapter{*}
\nociteart{*}
\nocitethes{*}
\nocitereview{*}
\nocitetextbook{*}
\nociteconf{*}
\nocitemisc{*}

\makecvtitle


%\section{Personal details}
\cventry{6-4-1983}{Born in Menden, Germany}{}{Menden, DE}{}{}

\section{Work experience}
\subsection{Academic work experience}


\cventry{2024--}{Full Professor\newline Holder of the Chair for Journalism Studies}{Vrije Universiteit Amsterdam}{Amsterdam, NL}{}{}
%\cventry{2020--}{Associate Professor (UHD)\newline Co-Director of the Digital Society Initiative}{Universiteit van Amsterdam}{Amsterdam, NL}{}{}
\cventry{2020--}{Associate Professor (UHD)}{Universiteit van Amsterdam}{Amsterdam, NL}{}{}

\cventry{2021--}{Adjunct Faculty (Professor II)}{Universitetet i Bergen}{Bergen, NO}{}{}

\cventry{2014--2020}{Assistant Professor (UD)}{Universiteit van Amsterdam}{Amsterdam, NL}{}{}

\cventry{2012--2014}{Lecturer (docent)}{Universiteit van Amsterdam}{Amsterdam, NL}{}{%Teaching bachelor and master courses on
%political communication as well as methods courses at the Department of Communication Science.%\newline{}%
\begin{itemize}
    \item Acquired the Dutch Basic Teaching Qualification (Basiskwalificatie Onderwijs) -- 7-10-2013
    \item Received 0.2 FTE research time grant for further qualification --  9/2013
\end{itemize}
}

\cventry{2009--2012}{PhD Candidate}{Universiteit van Amsterdam}{Amsterdam, NL}{}{
\begin{itemize}
\item Full-time employee as PhD candidate at the Amsterdam School of Communication Research
\item Teaching and thesis supervision
\end{itemize}
}



\cventry{2007--2009}{Student assistant}{Westf\"alische Wilhelms-Universit\"at}{M\"unster, DE}{}{
\begin{itemize}
\item Conducting quantitative and qualitative content analyses as student research assistant at the Department of Communication Science
\end{itemize}
}


\subsection{Journalistic work experience}
\cventry{2000--2009}{Freelance journalist for Westfalenpost and other regional publications}{Westfalenpost}{Menden, DE}{}{}

%\cventry{2005--2009}{Job title}{smart media solutions, Menden, Germany}{City}{}{Description}




\section{Education}
\cventry{2009--2013}{Doctoral degree (Dr.)}{Universiteit van Amsterdam}{Amsterdam, NL}{}{\begin{itemize}
\item Degree in Communication Science, defended at the Faculty of Social and Behavioural Sciences, with a thesis titled ``Following the News. Patterns of Online and Offline News Consumption''. %In the Netherlands, the German distinction between summa cum laude, magna cum laude, cum laude, and rite does not exist. 
\end{itemize}
}

\cventry{2006--2007}{Erasmus exchange}{Vrije Universiteit}{Amsterdam, NL}{}{}

\cventry{2003--2009}{Magister degree (M.A.) }{Westf\"alische Wilhems-Universit\"at}{M\"unster, DE}{}{
\begin{itemize}
\item Degree in Communication Science, minors in Dutch Studies and German Philology (grade: 1.11)
\item Thesis on ``New Papers for New Readers? Concepts, Profiles and Programs of the Dutch Papers nrc.next, De Pers and DAG'' (grade: 1.0)
\end{itemize}
}

%\section{Master thesis}
%\cvitem{title}{\emph{Title}}
%\cvitem{supervisors}{Supervisors}
%\cvitem{description}{Short thesis abstract}


\section{Languages}
%\cvitemwithcomment{German}{native language}{Comment}
%\cvitemwithcomment{English, Dutch}{near-native proficiency}{Comment}
%\cvitemwithcomment{French, Spanish, Norwegian, Latin}{Basic skills}{Comment}
\cvitem{German, Dutch, English}{native or near-native level proficiency}
\cvitem{French, Spanish, Norwegian, Latin}{basic skills}


\section{IT skills}
%\cvdoubleitem{statistical software}{R, STATA, SPSS, Amos}{software for qualitative analysis}{AtlasTI}
%\cvdoubleitem{programming in Python}{data retrieval, text analysis and natural language processing}{databases}{MySQL, MongoDB, ElasticSearch}
%\cvdoubleitem{markup languages}{LaTeX, HTML, CSS, XML}{system administration}{Linux, cloud computing, virtual machines}

\cvitem{Statistical software}{R, STATA, SPSS, Amos}
\cvitem{Programming}{profound Python knowledge, especially data retrieval, text analysis, natural language processing, classic machine learning and deep learning; basic knowledge of other programming languages such as C and JavaScript}
\cvitem{Databases}{MySQL, MongoDB, ElasticSearch}
\cvitem{Markup languages}{LaTeX, HTML, CSS, XML}
\cvitem{System administration}{Linux, cloud computing, virtual machines}
\cvitem{Software for qualitative analysis}{AtlasTI}
\cvitem{Office and graphics}{Prevalent office, DTP, graphic software}

%\section{Interests}
%\cvitem{hobby 1}{Description}
%\cvitem{hobby 2}{Description}
%\cvitem{hobby 3}{Description}


\section{Grants and awards}

\cvitem{OPINION: What are opinions? Integrating theory and methods for automatically analyzing opinionated communication}{I am one of initiators and main proposers of this COST action as well as management committee member and workgroup leaders; a cooperation between $>150$ scientists, running from 22-9-2022 to 21-9-2026.}


\cvitem{TWON: Twin of Online Networks}{I am the PI and coordinator of an ERC Horizon Grant (3M \euro{}), which runs from 1-4-2023 to 31-3-2026.}

\cvitem{Understanding climate polarization and depolarization dynamics}{For large-scale data annotation, Christel van Eck, Anne Kroon, and I received 5,000\euro{} from the Research Priority Area Communication in 2022}

\cvitem{Googling Politics?}{For a study on the use of search engines in news, I received, together with Marieke van Hoof, Judith M\"oller, and Corine Meppelink 5,000\euro{} from the Research Priority Area Communication in 2022}


\cvitem{NEWSFLOWS:  Modeling News Flows: How Feedback Loops Influence Citizens' Beliefs and Shape Societies}{I received a personal ERC Starting Grant (1.5M \euro{}), which runs from 1-1-2021 to 31-12-2025.}

\cvitem{Lowlands Science: ``Prik je eigen bubbel door!''} {In 2021, with a team of UvA and VU researchers, we were selected in a competitive procedure to collect tracking data during the Lowlands festival (postponed to 2022 due to Covid-19)}

\cvitem{LSRTMA: Large-Scale Real-Time Media Analysis}{In 2020, I received 35,700 \euro{} from the FNWI-HPC Funds Amsterdam to invest in hardware infrastructure (with Theo Araujo and Rens Vliegenthart).}

\cvitem{TwiXL: An infrastructure for cross-media research on public debates}{I am co-applicant (PI: Julia Noordegraaf) for a 978K \euro{} grant from the Platform Digitale Infrastructuur Social Science \& Humanities (PDI-SSH).}

\cvitem{MAPMOB: Mapping the mobile news diet}{For the development of a mobile lab to investigate mobile news consumption, I as PI received (with Judith Moeller, Felicia Loecherbach, and Wouter van Atteveldt) 5,000\euro{} from the Research Priority Area Communication in 2020 (cancelled due to Covid-19)}

\cvitem{Research Priority Area Communication funding}{For the project ``Effects of diversification of news and user agency in the context of algorithmic news recommenders'', Judith Moeller, Felicia Loecherbach, Wouter van Atteveldt, and I received 3,998\euro{}}

\cvitem{SHARENEWS: Predicting the Shareworthiness of ‘Real’ and ‘Fake’ News in Europe}{I am Principal Investigator of a project involving eleven researchers from four countries. Funded by a Social Media and Democracy Research Grant from the Social Science Research Council (SSRC), I received 50,000 USD in cash and a large Facebook dataset in kind.}

\cvitem{JEDS: Tracking the filter bubble}{I am co-applicant of the project ``Tracking the filter bubble'', funded with 244k \euro{}  in cash and 2.5 FTE working time of research engineers in kind. Funded by the Dutch eScience Center and the Netherlands Organisation for Scientific Research (NWO). Resulted in hiring a PhD candidate, who I supervise (in 2018).}

\cvitem{Competitive internal funding}{For the INCA project which I lead (``Infrastructure for Content Analysis'', \url{https://github.com/uvacw/inca}), we received in competitive funding rounds in total 24,250\euro{}  (2018: 2,500\euro{}, 2017: 16,750\euro{}, 2016: 5,000\euro{}).}

\cvitem{Research Priority Area Communication funding}{For the project ``Building a hype detector'', Iina Hellsten and I received 4,500\euro{}}

\cvitem{Research time for lecturers} {0.2 FTE research time funded while working as lecturer, granted in a competitive round for lecturers with the best research proposal. Resulted in publication of article \cite{VanKlingeren2020} (granted per 9/2013).}

\cvitem{Collegiality award}{I received an award for the most helpful colleague in the department (in 2017).}

\cvitem{Teaching award}{For my Research Master course ``Big Data and Automated Content Analyis'', I received an award for the best-evaluated methods course (in 2015).}

\cvitem{Travel grants}{I received more than 15,000\euro{} for conference travel (from 2010 onwards).}

\cvitem{Early PhD completion}{I received a bonus of 3,000\euro{} for early completion of my PhD (12/2012).}




\section{Leadership}
\begin{itemize}
\item Management Committee member, Country Representative, and Work Group Leader for the COST action OPINION, an international cooperation of $>100$ scientists
\item Principal Investigator and Coordinator of a ERC Horizon Grant, involving 8 partners from 4 countries 
\item Principal Investigator of an ERC Starting Grant, involving three PhD candidates, one PostDoc, and multiple student assistants.
\item Principal Investigator of the SHARENEWS project, involving eleven researchers from four countries.
\item Leader of the INCA project (``Infrastructure for Content Analysis'', \url{https://github.com/uvacw/inca}), in which we develop and maintain an infrastructure for automated content analysis, used by researchers across the Department of Communication Science. I continuously supervise student assistants, interns, and scientific staff that contribute to the project. 
\end{itemize}


\section{Current research projects}

\cventry{2023--2026}{ERC Horizon}{TWON: Twin of Online Networks}{}{}{I coordinate an large-scale project with 10 partners from 5 countries, in which we aim to build a digital twin for the study of online platforms.}


\cventry{2021--2025}{ERC Starting Grant}{NEWSFLOWS:  Modeling News Flows: How Feedback Loops Influence Citizens' Beliefs and Shape Societies}{}{}{In this large-scale project, we use computational methods and field experiments to investigate feedback loops in news dissemination (\url{https://newsflows.eu}).}

\cventry{2021--}{PDI-SSH}{TwiXL: An infrastructure for cross-media research on public debates}{}{}{In this project, we make digital media content accessible for sustainable large-scale research projects.}

\cventry{2017--}{Internal project; project with Volkskrant, NL; project with LMU M\"unchen, DE}{Use of, attitudes towards, and effects of news recommender systems}{}{}{We investigated (with Neil Thurman, LMU, Judith M\"oller, UvA, Natali Helberger, UvA) how media users evaluate the use of algorithmic news recommendations; with Felicia L\"ocherbach, VU, NL I developed a platform to test their effects, and with Judith M\"oller, Natali Helberger, Bram van Es (all UvA, NL), I investigated in cooperation with a large Dutch newspaper the impact of news recommendations on content diversity.}

\cventry{2014--}{Internal project}{Infrastructure for content analysis}{}{}{Development and maintenance of software and hardware infrastructure for automated content analysis.}

\cventry{2018--2022}{Projects funded by NWO (Dutch Science Foundaton) and ASCoR}{Online news use}{}{}{Together with two PhD students that I co-supervise (see sections \textit{PhD supervision} and \textit{grants and awards}), I analyze the use of online news use.}



\section{Completed research projects}

\cventry{2019--2020}{Social Science Research Council (SSRC): Social Media and Democracy Research Grant}{SHARENEWS: Predicting the Shareworthiness of `Real' and `Fake' News in Europe}{}{}{In this project, we use a large-scale dataset of Facebook URL shares to investigate which factors explains the shareworthiness of news on social media, spanning four countries and very different kinds of news.}

\cventry{2018--2020}{Internal project}{Detecting and analyzing news events}{}{}{In this project, I try to develop a method to automatically identify ``news events'' in large corpora of news articles in order to answer questions about intermedia agenda-setting effects, news diffusion, and news diversity.}

\cventry{2012--2019}{Project with U of Missouri, US, and U Luzern, CH}{The credibility of credibility measures}{}{}{The project with Lea Hellm\"uller and Anina Hanimann aims at creating an extensive review of the use of credibility measures in the communication science literature.}


\cventry{2018}{Project for UvA Research Priority Area Communication}{Building a hype-detector}{}{}{In this project with Iina Hellsten, we explored which techniques to use to detect media hypes. Funded by UvA RPA Communication with 4.500\euro{}.}

\cventry{2016--2018}{Internal project}{The State of automated content analysis}{}{}{In this project, we mapped the state of the art of automated content analysis and developed suggestions for further methodological development.}

\cventry{2015--2018}{Interdisciplinary project}{Personalised Communiation}{}{}{I participated in this large interdisciplinary project (PIs: Natali Helberger and Claes de Vreese) with 0.2 FTE for three years. Using both tracking data and survey data, I focused on the analysis of personalised news use.}

\cventry{2016--2017}{Internal project}{Economic news}{}{}{With several colleagues, I looked into coverage about companies, including the relationship with stock exchange rates.}

\cventry{2015--2017}{Internal project}{News values and news sharing}{}{}{In this project, I analyzed news values in online and offline news, and how these are related to news sharing.}

\cventry{2015}{Internal project}{Incivility in online comments}{}{}{Together with Linda Bos, I investigated the deliberativeness and incivility of comments on online news sites.}

\cventry{2013--2018}{Project with TU Dortmund, DE, FU Berlin, DE, and RU Nijmegen, NL}{Twitter and the public sphere}{}{}{I investigated debates on Twitter in Brazil and the Netherlands, together with Mariella Bastian, Débora Maria Moura Medeiros, Judith M\"oller, and Marijn van Klingeren.}

\cventry{2013--2015}{Project with U of Haifa, IL and RU Nijmegen, NL}{Selective exposure in a multi-party system}{}{}{In this project with Yariv Tsfati, University of Haifa, and Marijn van Klingeren, Radboud Universiteit Nijmegen, I conducted an online experiment on the effects of selective exposure in the Dutch multi-party system. The project was funded in kind with 0.2 FTE research time by the Graduate School of Communication and with $\approx$ 2,000\euro{} in cash for data collection costs by the University of Haifa.}

\cventry{2012--2017}{Internal project}{Second screen usage}{}{}{In this project, I investigated how Twitter content reflects television content. The second part was done together with Mark Boukes.}

\cventry{2009--2013}{PhD project}{Patterns of online and offline news consumption}{}{}{In this project funded by the Amsterdam School of Communication Research and the Dutch Press Fund, I conducted survey research on news media use in the Netherlands and Austria.}




\section{Supervision}
\subsection{PhD supervision}
\cvitemwithcomment{Roeland Dub\`el}{``Trust challenges facing journalism''}{2024-2028}
{(co-supervision with Mark Boukes and Sandra Jacobs)} \newline

\cvitemwithcomment{Rupert Kiddle}{``Feedback loops and beliefs''}{2022-2025}
(co-supervision with Anne Kroon and Kasper Welbers) \newline

%\cvitemwithcomment{Khadiga Mahmoud Abdalla Seddik}{``The double-edged sword of news recommenders''}{2022-2026}
\cvitemwithcomment{Khadiga Seddik}{``The double-edged sword of news recommenders''}{2022-2026}
(at UiB, co-supervision with Erik Knudsen and Christoph Trattner) \newline

%\cvitemwithcomment{Zilin Lin}{``Feedback loops in the interplay of manual news sharing and algorithmic recommendations''}{2021-2024}
\cvitemwithcomment{Zilin Lin}{``Feedback loops in the interplay of sharing and newsrecommendation''}{2021-2024}
(co-supervision with Susan Vermeer and Kasper Welbers) \newline

\cvitemwithcomment{M\'onika Simon}{``Feedback loops in cross-domain flows''}{2021-2024}
(co-supervision with Anne Kroon and Kasper Welbers) \newline

\cvitemwithcomment{Marieke van Hoof}{``Polarizing Issue Publics''}{2020-2024}
(co-supervision with Judith M\"oller and Corine Meppelink) \newline

\cvitemwithcomment{Felicia L\"ocherbach}{``Tracking the filter bubble''}{2018-2023}
(co-supervision with Judith M\"oller and Wouter van Atteveldt; defended on 6-6-2023) \newline

\cvitemwithcomment{Susan Vermeer}{``News for you!''} {2018-2021}
(co-supervision with Sanne Kruikemeier and Claes de Vreese; defended on 19-11-2021) \newline

\cvitemwithcomment{Tom Dobber}{``Extent and consequences of political behavioral targeting''}{2016-2020}
(co-supervison with Natali Helberger and Claes de Vreese; defended on 30-6-2020) \newline



\subsection{Membership in PhD dissertation committees}
\cvitemwithcomment{Moritz Laurer}{Vrije Universiteit Amsterdam}{4-10-2024}
``Language Models as Measurement Tools: Using Instruction-Based Models to Increase Validity, Robustness and Data Efficiency'', supervised by Wouter van Atteveldt, Kasper Welbers, and Andreu Casas \newline

\cvitemwithcomment{Erik de Vries}{Universitetet i Stavanger}{6-6-2024}
``Varying bits: A computational perspective on news diversity and political parallelism'', supervised by Gunnar Thesen, Rens Vliegenthart, and Gijs Schumacher \newline

\cvitemwithcomment{Carlos Brenes Peralta}{Universiteit van Amsterdam}{ 19-9-2017}
``Two sides to every story'', supervised by Claes de Vreese, Magdalena Wojcieszak, and Yphtach Lelkes \newline


\subsection{Bachelor and Master supervision}
I supervised $>100$ Bachelor and Master theses and have read hundreds of theses as a second reader.



\section{Service to the discipline}
\subsection{Ongoing activities}
\cventry{2024--} {``Genderneutrale formulering van wetgeving'' (Gender-neutral formulations in legislation by Wetenscappelijk onderzoek- en datacentrum (WODC), Den Haag}{Advisory Board member}{}{}{}


\cventry{2022}{Jornalism \& Mass Communication Quartletly}{Editorial Board member}{}{}{}

\cventry{2021--} {Stichting Computational Communication Research}{Treasurer}{}{}{}

\cventry{2021--2025} {NEWSREC: The Double-edged Sword of News Recommenders' Impact on Democracy, University of Bergen, Norway}{Core group member}{}{}{}

\cventry{2021--} {``Academische werkplaats desinformatie'' (Academic workshop disinformation, thinktank of the Municipality of Amsterdam}{Member}{}{}{}

\cventry{2021--2023} {``Betwiste informatie in het coronadebat'' (Contested information in the corona debate by Sociaal en Cultureel Planbureau (SCP), Den Haag}{Advisory Board member}{}{}{}

\cventry{2022--2024} {POLTRACK ``Political polarization and individualized online information environments: A longitudinal tracking study'' by Hans-Bredow-Institut, GESIS, U Konstanz, U Bremen}{Advisory Board member}{}{}{}

\cventry{2021--2026}{PDI-SSH project ``Capture and Analysis Tools for Social Media Research (CAT4SMR)''}{Advisory Board member}{}{}{}

\cventry{2021--}{``AI + Journalism Project'' at Institute of Communication Research at Seoul National University} {Advisory Board member}{}{}{}

\cventry{2021--present}{Digital Jornalism}{Editorial Board member}{}{}{}

\cventry{2018--present}{Computational Communication Research}{Founding associate editor}{}{}{}

\cventry{2015--present}{Advising College and Gradudate School regarding the methods curriculum}{Member of Methods Committee}{}{}{}

\cventry{continously}{Regular reviewer of manuscripts for conferences and journals, including:}{Ad-hoc reviewer and/or programme committee member}{}{}{}
Journal of Communication, Social Science Computer
Review, New Media \& Society, Mass Communication and Society,
Communication Methods and Measures, Medien \&
Kommunikationswissenschaft, Tijdschrift voor
Communicatiewetenschap, International Journal of Public Opinion
Research, International Journal of Communication,
Communications: The European Journal of Communication
Research, Journal of Broadcasting and Electronic Media, Mobile
Media \& Communication, Journal of Computer-Mediated
Communication, First Monday, Nordicom Review, Information Processing \& Management,
Journalism \& Mass Communicaiton Quarterly, Internet Research;
ICA, ECREA, WAPOR, Etmaal van de Communicatiewetenschap.
International Conference on the Web and Social Media (ICWSM),
Internaternational Conference on Computational Social Science (IC2S2),
ACM Conference on User Modeling, Adaptation and Personalization (UMAP).

\emph{Also reviewing for several funding agencies, including the Czech
  Science Foundation, the Flemish Science Foundation, and the Swiss National Science Foundation.}



\subsection{Past activities}

\cventry{2020}{``Social Science and Humanities Covid-19 Expertise Potal (https://ssh-covid19.nl/)''}{Team member}{}{}{}

\cventry{2017--2020}{Journalism, Media and Globalisation (Erasmus Mundus Master's joint degree)}{Vice Chair of Programme Committee (Opleidingscomissie)}{}{}{}

\cventry{2017--2020}{Second reader to guarantee quality of master theses}{Member of Committee ``Thesis Quality Master''}{}{}{}

\cventry{2016}{Developing curriculum for new Master program at the Faculty of Science}{Member of the curriculum committee MSc Data Science}{}{}{}

\cventry{2015--2016}{JongUvA -- committee to organize social events for young colleagues}{Member of Social Commitee}{}{}{}






\section{Teaching}
%\cvitemwithcomment{bla}{natirrerjerje}{Comment}
%\cvitemwithcomment{bla, Dutch}{nerejerjjiency}{Comment}
%\cvitemwithcomment{Frenchblan}{Breueruruls}{Comment}





\cvitem{Big Data and Automated Content Analysis Part I+II}{Research Master methods course. Taught every academic year since 2018/2019, in semester 2.}

\cvitem{Big Data and Automated Content Analysis}{Research Master methods course. Taught every academic year since 2013/2014, in semester 2.}

\cvitem{Data Journalism (previously: Innovating Journalism)}{Master course on data journalism. Co-teaching with Penny Sheets every academic year since 2018/19, in semester 1.}

\cvitem{Journalism and the Media}{Master specialization seminar. Co-teaching with Penny sheets every academic year since 2017/18, semester 2.}

\cvitem{Journalistic Product}{Practical course to turn Master's thesis into a journalistic product. Taught in academic year 2016/17, semester 2, and academic year 2017/18, semester 2.}

\cvitem{Media Ethics}{MOOC (Massive Open Online Course). Developed with Rutger de Graaf, Elgin Blankwater, Annemarie van Oosten, Sandra Jacobs, Lotte Salome. Continously available online.}

\cvitem{Introduction to Research Methods}{Seminar group in the pre-master program. Taught in academic year 2014/15, semester 1.}

\cvitem{Hot topics in political communication}{Bachelor graduation project (Afstudeerproject). Co-teaching with Linda Bos in academic year 2014/15, semester 1.}

\cvitem{Graduation project Excellence track (Afstudeerproject Excellentietraject)}{Bachelor graduation project for excellent students. Taught in academic year 2013/14, semester 2 and academic year 2014/15, semester 2.}

\cvitem{Citizens and Public Opinion}{Master specialization seminar. Co-teaching with Anouk van Drunen and Regula H\"anglli in academic year 2012/13, semester 2.}

\cvitem{Journalism and the Media}{Master specialization seminar. Co-teaching with Richard van der Wurff in academic year 2012/13, semester 1.}

\cvitem{New Media, New Politics?}{Master elective. Taught in academic year 2012/13, semester 1 and academic year 2012/13, semester 2.}

\cvitem{Domain Module Political Communication and Journalism}{Bachelor seminar groups (Werkgroepen Domeinmodule politieke communicatie en journalistiek). Two groups taught in academic year 2012/13, semester 1; two groups taught in academic year 2015/16, semester 1.}

\cvitem{Dead Trees and Digital Citizens: News and Journalism in an Online Era}{Bachelor graduation seminar (Afstudeerseminar Dode bomen en digitale burgers: Nieuws en journalistiek in het online tijdperk). Co-teaching with Tom Bakker in academic year 2011/12, semester 2; taught as graduation project (afstudeerproject) in academic year 2012/13, semester 2 and academic year 2013/14, semester 1.}

\cvitem{Introduction to Communication Science} {Bachelor seminar group (Werkgroep Inleiding Communicatiewetenschap). Taught in academic year 2010/11, semester 1.}


\section{Invited talks and workshops (selection)}

\cventry{03-2023}{Invited Talk``News and Political Information in the Digital Society -- The Role of Human and Algorithmic Feedback Loops''}{National University of Singapore}{}{}{}

\cventry{12-2022}{Keynote ``News and Political Information in the Digital Society -- The Role of Human and Algorithmic Feedback Loops''}{EMERGE 2022: Forum on the Future of AI Driven Humanity \& International Conference Digital Society Now, Belgrado, Serbia}{}{}{}

\cventry{9-2022}{One-week intensive course ``Introduction to Machine Learning for Text Analysis with Python'' (with Anne Kroon)}{GESIS Fall Seminar, Mannheim, Germany}{}{}{}


\cventry{12-2021}{Invited talk ``Modeling News Flows: How Feedback Loops Influence Citizens' Beliefs and Shape Societies'' (MediaFutures Seminar)}{University of Bergen, NO}{}{}{}

\cventry{9-2021}{Invited panelist ``News Personalization in the Age of Fake News and Polarization''}{9th International Workshop on News Recommendation and Analytics (INRA 2021) in conjunction with RecSys 2021}{}{}{}


\cventry{9-2021}{One-week intensive course ``Introduction to Machine Learning in Python'' (with Anne Kroon)}{GESIS Fall Seminar, Cologne, Germany (online due to Corona}{}{}{}

\cventry{3-2021}{Invited talk ``Modeling News Flows: How Feedback Loops Influence Citizens' Beliefs and Shape Societies'' (Distinguished Speaker Series)}{IKMZ, University of Zurich, CH}{}{}{}


\cventry{3-2020}{One-week intensive course ``Introduction to Machine Learning in Python'' (with Anne Kroon)}{GESIS Spring Seminar, Cologne, Germany}{}{}{}

\cventry{9/10-2019}{Invited speaker at lunch seminar and invited visiting researcher}{University of Bergen, NO}{}{}{}

\cventry{8-2018}{Invited speaker and project supervisor at the Summer School on Methods for Computational Social Science by GESIS and Volkswagen Stiftung}{CSS Summer School, Los Angeles, USA}{}{}{}

\cventry{10-2017}{Invited speaker at symposium ``Social Media and Democracy: New Challenges for Political Communication Research''}{University of Copenhagen, DK \& University of Lund, SE}{}{}{}

\cventry{9-2017}{Four-day workshop Automated Content Analysis with Python}{Universiteit Antwerpen, BE}{}{}{}

\cventry{14-9-2017}{Invited speaker at Workshop Medien in Konflikten.}{Schader-Forum, Darmstadt, DE}{}{}{}

\cventry{5-2017}{Two-day workshop Automated Content Analysis with Python}{Radboud Universiteit, Nijmegen, NL}{}{}{}

\cventry{10-4-2017}{Invited presentation:``Filter bubbles are overrated''}{Vrije Universiteit, Amsterdam, NL}{}{}{}

\cventry{27-3-2017}{Participant in a panel discussion on filter bubbles and algorithmic news selection}{Pakhuis de Zwijger, ``Sign of Time'' series, Amsterdam, NL}{}{}{}

\cventry{2/2017}{Two-day workshop Automated Content Analysis with Python}{Amsterdam Institute for Social Science Research, Universiteit van Amsterdam}{}{}{}

\cventry{11-2016}{Invited speaker at Future of Journalism Seminar ``Analysing Social Media Data: Methodologies \& Case Studies''}{Dublin City University, IRE}{}{}{}

\cventry{12-5-2016}{Session at PhD training workshop: ''Data analysis in the digital age''
%��”Qualitative research methods in the digital era: How
%interviews have been re-shaped by technology?
}{Geneva,
CH}{}{}{} 
 
\cventry{4-2016}{Invited participant at expert workshop ``Analysis, Interpretation and Benefit of User-Generated Data: Computer Science Meets Communication Studies''}{Schloss Dagstuhl, DE}{}{}{}

\cventry{1-3-2016}{Invited talk: ``Inductive automated frame analysis''
%.  at “Expert meeting on automatic text analysis”,
}{Utrecht School of Governance, Utrecht
University, NL}{}{}{}
 
 \cventry{24-2-2016}{Invited talk: ``Setting up an infrastructure for large-scale automated content analysis''}{Computational Social Science Meetup, Freie Universität Berlin, DE}{}{}{}

\cventry{8-12-2015}{Guest lecture: ``Big Data \& network analysis'' %department of Political Science at RU Nijmegen
}{Radboud Universiteit, Nijmegen, NL}{}{}{}

\cventry{1-12-2015}{Invited talk at pre-symposium workshop: ``From word frequencies to topic modeling: Applying automated content analysis techniques to short social media messages''  }{GESIS Computational Social Science Winter Symposium, Cologne, DE}{}{}{}

\cventry{27-11-2015}{Invited lecture: ``Packing and unpacking the bag of words: Introducing a toolkit for inductive automated frame analysis.'' }{IdeaLabsSymposium, %“Social media: incubators of a renewed news media landscape?”
KU Leuven, BE}{}{}{}

 \cventry{22-9-2015}{Invited talk: ``Meer dan sentimentscores: Inzichten destilleren uit een enorme hoeveelheid data''}{Nederlandstalig Platform voor Surveyonderzoek (NPSO), Amsterdam, NL}{}{}{}

\cventry{8-10-2014 }{Workshop ``Python in the Social Sciences''}{Utrecht Data School, Utrecht University, NL}{}{}{}

\cventry{5-3-2014}{Workshop ``Python in the Social Sciences''}{Coding Culture, Utrecht, NL}{}{}{}

\cventry{3-2014}{Hands-on-workshop ``Analyzing Big (Twitter) Data''}{Department of Communication Science, Universiteit van Amsterdam, NL}{}{}{}
 


%\section{Extra 1}
%\cvlistitem{Item 1}
%\cvlistitem{Item 2}
%\cvlistitem{Item 3. This item is particularly long and therefore normally spans over several lines. Did you notice the indentation when the line wraps?}

%\section{Extra 2}
%\cvlistdoubleitem{Item 1}{Item 4}
%\cvlistdoubleitem{Item 2}{Item 5\cite{book1}}
%\cvlistdoubleitem{Item 3}{Item 6. Like item 3 in the single column list before, this item is particularly long to wrap over several lines.}

%\section{References}
%\begin{cvcolumns}
%  \cvcolumn{Category 1}{\begin{itemize}\item Person 1\item Person 2\item Person 3\end{itemize}}
%  \cvcolumn{Category 2}{Amongst others:\begin{itemize}\item Person 1, and\item Person 2\end{itemize}(more upon request)}
%  \cvcolumn[0.5]{All the rest \& some more}{\textit{That} person, and %\textbf{those} also (all available upon request).}
% \end{cvcolumns}


\section{Publications}


%\iffalse

\subsection{Manuscripts under review}

\makeatletter

% als we als nummering 1. ipv [1] willen
%\renewcommand\@biblabel[1]{#1.}


\long\def\thebibliography#1{%
  %\section*{\refname}%
  %\@mkboth{\MakeUppercase\refname}{\MakeUppercase\refname}
  \settowidth{\dimen0}{\@biblabel{#1}+5}%  ADD OFFSET (5 in dit geval, gewoon gegokt) zodat het nog steeds goed is uitgelijnd
  \setlength{\dimen2}{\dimen0}%
  \addtolength{\dimen2}{\labelsep}
  \sloppy
  \clubpenalty 4000 
  \@clubpenalty 
  \clubpenalty 
  \widowpenalty 4000
  \sfcode `\.\@m
  \renewcommand{\labelenumi}{\@biblabel{R \theenumi}} % labels like [3], [2], [1]
  \begin{etaremune}[labelwidth=\dimen0,leftmargin=\dimen2]\@openbib@code
}
\def\endthebibliography{\end{etaremune}}
\def\@bibitem#1{%
  \item \if@filesw\immediate\write\@auxout{\string\bibcite{#1}{\the\value{enumi}}}\fi\ignorespaces
}
\makeatother


\bibliographystylereview{unsrtdindamian}  
\bibliographyreview{review}                  

%\fi




\subsection{Peer-reviewed articles in journals and proceedings}

\makeatletter
\long\def\thebibliography#1{%
  %\section*{\refname}%
  %\@mkboth{\MakeUppercase\refname}{\MakeUppercase\refname}
  \settowidth{\dimen0}{\@biblabel{#1}+5}%  ADD OFFSET (5 in dit geval, gewoon gegokt) zodat het nog steeds goed is uitgelijnd
  \setlength{\dimen2}{\dimen0}%
  \addtolength{\dimen2}{\labelsep}
  \sloppy
  \clubpenalty 4000 
  \@clubpenalty 
  \clubpenalty 
  \widowpenalty 4000
  \sfcode `\.\@m
  \renewcommand{\labelenumi}{\@biblabel{A \theenumi}} % labels like [3], [2], [1]
  \begin{etaremune}[labelwidth=\dimen0,leftmargin=\dimen2]\@openbib@code
}
\def\endthebibliography{\end{etaremune}}
\def\@bibitem#1{%
  \item \if@filesw\immediate\write\@auxout{\string\bibcite{#1}{\the\value{enumi}}}\fi\ignorespaces
}
\makeatother

\bibliographystyleart{unsrtdindamian}  
\bibliographyart{articles}   





\subsection{Theses}


\makeatletter
\long\def\thebibliography#1{%
  %\section*{\refname}%
  %\@mkboth{\MakeUppercase\refname}{\MakeUppercase\refname}
  \settowidth{\dimen0}{\@biblabel{#1}+5}%  ADD OFFSET (5 in dit geval, gewoon gegokt) zodat het nog steeds goed is uitgelijnd
  \setlength{\dimen2}{\dimen0}%
  \addtolength{\dimen2}{\labelsep}
  \sloppy
  \clubpenalty 4000 
  \@clubpenalty 
  \clubpenalty 
  \widowpenalty 4000
  \sfcode `\.\@m
  \renewcommand{\labelenumi}{\@biblabel{T \theenumi}} % labels like [3], [2], [1]
  \begin{etaremune}[labelwidth=\dimen0,leftmargin=\dimen2]\@openbib@code
}
\def\endthebibliography{\end{etaremune}}
\def\@bibitem#1{%
  \item \if@filesw\immediate\write\@auxout{\string\bibcite{#1}{\the\value{enumi}}}\fi\ignorespaces
}
\makeatother



\bibliographystylethes{unsrtdindamian}    % of unsrt
\bibliographythes{theses}                  


\subsection{Textbooks}

\makeatletter
\long\def\thebibliography#1{%
  %\section*{\refname}%
  %\@mkboth{\MakeUppercase\refname}{\MakeUppercase\refname}
  \settowidth{\dimen0}{\@biblabel{#1}+5}%  ADD OFFSET (5 in dit geval, gewoon gegokt) zodat het nog steeds goed is uitgelijnd
  \setlength{\dimen2}{\dimen0}%
  \addtolength{\dimen2}{\labelsep}
  \sloppy
  \clubpenalty 4000 
  \@clubpenalty 
  \clubpenalty 
  \widowpenalty 4000
  \sfcode `\.\@m
  \renewcommand{\labelenumi}{\@biblabel{TB \theenumi}} % labels like [3], [2], [1]
  \begin{etaremune}[labelwidth=\dimen0,leftmargin=\dimen2]\@openbib@code
}
\def\endthebibliography{\end{etaremune}}
\def\@bibitem#1{%
  \item \if@filesw\immediate\write\@auxout{\string\bibcite{#1}{\the\value{enumi}}}\fi\ignorespaces
}
\makeatother



\bibliographystyletextbook{unsrtdindamian}    % of unsrt
\bibliographytextbook{textbooks}                  





\subsection{Book chapters}


\makeatletter
\long\def\thebibliography#1{%
  %\section*{\refname}%
  %\@mkboth{\MakeUppercase\refname}{\MakeUppercase\refname}
  \settowidth{\dimen0}{\@biblabel{#1}+5}%  ADD OFFSET (5 in dit geval, gewoon gegokt) zodat het nog steeds goed is uitgelijnd
  \setlength{\dimen2}{\dimen0}%
  \addtolength{\dimen2}{\labelsep}
  \sloppy
  \clubpenalty 4000 
  \@clubpenalty 
  \clubpenalty 
  \widowpenalty 4000
  \sfcode `\.\@m
  \renewcommand{\labelenumi}{\@biblabel{C \theenumi}} % labels like [3], [2], [1]
  \begin{etaremune}[labelwidth=\dimen0,leftmargin=\dimen2]\@openbib@code
}
\def\endthebibliography{\end{etaremune}}
\def\@bibitem#1{%
  \item \if@filesw\immediate\write\@auxout{\string\bibcite{#1}{\the\value{enumi}}}\fi\ignorespaces
}
\makeatother


\bibliographystylechapter{unsrtdindamian}   %ofunsrtdindamian dan krijg je o.a. ook doi 
\bibliographychapter{bookchapters}                  





\subsection{Other publications (selection)}

\makeatletter
\long\def\thebibliography#1{%
  %\section*{\refname}%
  %\@mkboth{\MakeUppercase\refname}{\MakeUppercase\refname}
  \settowidth{\dimen0}{\@biblabel{#1}+5}%  ADD OFFSET (5 in dit geval, gewoon gegokt) zodat het nog steeds goed is uitgelijnd
  \setlength{\dimen2}{\dimen0}%
  \addtolength{\dimen2}{\labelsep}
  \sloppy
  \clubpenalty 4000 
  \@clubpenalty 
  \clubpenalty 
  \widowpenalty 4000
  \sfcode `\.\@m
  \renewcommand{\labelenumi}{\@biblabel{O \theenumi}} % labels like [3], [2], [1]
  \begin{etaremune}[labelwidth=\dimen0,leftmargin=\dimen2]\@openbib@code
}
\def\endthebibliography{\end{etaremune}}
\def\@bibitem#1{%
  \item \if@filesw\immediate\write\@auxout{\string\bibcite{#1}{\the\value{enumi}}}\fi\ignorespaces
}
\makeatother


\bibliographystylemisc{unsrtdindamian}  
\bibliographymisc{misc}                  







\subsection{Conference presentations and posters (non-archival)}


\makeatletter
\long\def\thebibliography#1{%
  %\section*{\refname}%
  %\@mkboth{\MakeUppercase\refname}{\MakeUppercase\refname}
  \settowidth{\dimen0}{\@biblabel{#1}+5}%  ADD OFFSET (5 in dit geval, gewoon gegokt) zodat het nog steeds goed is uitgelijnd
  \setlength{\dimen2}{\dimen0}%
  \addtolength{\dimen2}{\labelsep}
  \sloppy
  \clubpenalty 4000 
  \@clubpenalty 
  \clubpenalty 
  \widowpenalty 4000
  \sfcode `\.\@m
  \renewcommand{\labelenumi}{\@biblabel{P \theenumi}} % labels like [3], [2], [1]
  \begin{etaremune}[labelwidth=\dimen0,leftmargin=\dimen2]\@openbib@code
}
\def\endthebibliography{\end{etaremune}}
\def\@bibitem#1{%
  \item \if@filesw\immediate\write\@auxout{\string\bibcite{#1}{\the\value{enumi}}}\fi\ignorespaces
}
\makeatother


\bibliographystyleconf{unsrtdindamian}  
\bibliographyconf{conferences}                  






\clearpage








\end{document}

