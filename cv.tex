% arara: lmkclean
% arara: clean: { files: [art.bbl, art.aux, thes.bbl, thes.aux, conf.bbl, conf.aux, textbook.bbl, textbook.aux., chapter.bbl, chapter.aux, review.bbl, review.aux, misc.bbl, misc.aux] }
% arara: pdflatex
% arara: bibtex: { files: [ art,thes,conf,textbook,chapter,review,misc ] }
% arara: pdflatex
% arara: pdflatex

\documentclass[11pt,a4paper,sans]{moderncv} 
\moderncvstyle{banking}                             % style options are 'casual' (default), 'classic', 'oldstyle' and 'banking'
\moderncvcolor{blue}                               % color options 'blue' (default), 'orange', 'green', 'red', 'purple', 'grey' and 'black'
\usepackage[utf8]{inputenc}
\usepackage[scale=0.75]{geometry}
%\setlength{\hintscolumnwidth}{3cm}                % if you want to change the width of the column with the dates
%\setlength{\makecvtitlenamewidth}{10cm}           % for the 'classic' style, if you want to f\textsc{}orce the width allocated to your name and avoid line breaks. be careful though, the length is normally calculated to avoid any overlap with your personal info; use this at your own typographical risks...




% OPEN VRAGEN
% GEBOORTEJAAR MEENMEN?




\usepackage{todonotes}

% personal data
\name{Damian}{Trilling}
\title{CV}
\address{Boeroestraat 44}{1095VS AMSTERDAM}{The Netherlands}
\phone[mobile]{+31~6 48 13 35 76}  
\email{d.c.trilling@uva.nl}
\homepage{www.damiantrilling.net} 
\social[twitter]{damian0604}
\social[github]{damian0604}      
%\extrainfo{additional information}       
%\photo[64pt][0.4pt]{bestandsnaam}


% zie https://tex.stackexchange.com/questions/161897/moderncv-newline-within-a-cventry-in-the-banking-style
% maakt multi-line cventry mogelijk
% of 13cm de juiste keuze is moet nog blijken
\renewcommand*{\cventry}[7][.25em]{
	\begin{tabular*}{\textwidth}{p{13cm}@{\extracolsep{\fill}}r}%
		{\bfseries #4} & {\bfseries #5} \\%
		{\itshape #3\ifthenelse{\equal{#6}{}}{}{, #6}} & {\itshape #2}\\%
	\end{tabular*}%
	\ifx&#7&%
	\else{\\\vbox{\small#7}}\fi%
	\par\addvspace{#1}}



% to show numerical labels in the bibliography (default is to show no labels); only useful if you make citations in your resume
%\makeatletter
%\renewcommand*{\bibliographyitemlabel}{\@biblabel{\arabic{enumiv}}}
%\makeatother
%\renewcommand*{\bibliographyitemlabel}{[\arabic{enumiv}]}% CONSIDER REPLACING THE ABOVE BY THIS




\usepackage{etaremune}
% bibliography with mutiple entries
\usepackage{multibib}
\newcites{art,thes,conf,textbook,chapter,review,misc}{{Articles},{Theses},{Conference Presentationn and Posters},{Textbooks},{Book chapters},{Manuscripts under review},{Other publications}}




\begin{document}

% generate all citations now already to avoid messing with the order in case I cite somehwere in the manuscript
\nocitechapter{*}
\nociteart{*}
\nocitethes{*}
\nocitereview{*}
\nocitetextbook{*}
\nociteconf{*}
\nocitemisc{*}

\makecvtitle


\section{Work experience}
\subsection{Academic work experience}


\cventry{2014--present}{Assistant Professor (UD), tenured}{Universiteit van Amsterdam}{Amsterdam}{}{
%\begin{itemize}
%    \item  ehkelhk
%    \item ejgekjhe
%    \item PhD supervison
%    \item meer dan 100 scripties
%    \item .........
%\end{itemize}
}


\cventry{2012--2014}{Lecturer (docent)}{Universiteit van Amsterdam}{Amsterdam}{}{%Teaching bachelor and master courses on
%political communication as well as methods courses at the Department of Communication Science.%\newline{}%
\begin{itemize}
    \item Acquired the Dutch Basic Teaching Qualification (Basiskwalificatie Onderwijs) -- 7-10-2013
    \item Received 0.2 FTE research time grant for further qualification --  9/2013
\end{itemize}
}

\cventry{2009--2012}{PhD Candidate}{Universiteit van Amsterdam}{Amsterdam}{}{
\begin{itemize}
\item Full-time employee as PhD candidate at the Amsterdam School of Communication Research
\item Teaching and thesis supervision
\end{itemize}
}



\cventry{2007--2009}{Student assistant}{Westf\"alische Wilhelms-Universit\"at}{M\"unster}{}{
\begin{itemize}
\item Conducting quantitative and qualitative content analyses as student research assistant at the Department of Communication Science.
\end{itemize}
}


\subsection{Journalistic work experience}
\cventry{2000–-2009}{Freelance journalist for Westfalenpost and other regional publications}{Westfalenpost}{Menden}{}{}





\section{Education}
\cventry{2009--2013}{Doctoral degree (Dr.)}{Universiteit van Amsterdam }{Amsterdam}{}{\begin{itemize}
\item Degree in Communication Science with a thesis titled “Following the News. Patterns of Online and Offline News Consumption”. 
\end{itemize}
}

\cventry{2006--2007}{Erasmus exchange}{Vrije Universiteit}{Amsterdam}{}{}

\cventry{2003--2009}{Magister degree (M.A.) }{Westf\"alische Wilhems-Universit\"at}{M\"unster}{}{
\begin{itemize}
\item Degree in Communication Science, minors in Dutch Studies and German Philology (grade: 1.11)
\item Thesis on ``New Papers for New Readers? Concepts, Profiles and Programs of the Dutch Papers nrc.next, De Pers and DAG'' (grade: 1.0)
\end{itemize}
}

%\section{Master thesis}
%\cvitem{title}{\emph{Title}}
%\cvitem{supervisors}{Supervisors}
%\cvitem{description}{Short thesis abstract}


\section{Languages}
%\cvitemwithcomment{German}{native language}{Comment}
%\cvitemwithcomment{English, Dutch}{near-native proficiency}{Comment}
%\cvitemwithcomment{French, Spanish, Norwegian, Latin}{Basic skills}{Comment}
\cvitem{German, Dutch}{native-level proficiency}
\cvitem{English}{near-native-level proficiency}
\cvitem{French, Spanish, Norwegian, Latin}{basic skills}


\section{IT skills}

\cvitem{Statistical software}{R, STATA, SPSS, Amos}
\cvitem{Programming in Python}{data retrieval, text analysis and
natural language processing}
\cvitem{Databases}{MySQL, MongoDB, ElasticSearch}
\cvitem{Markup languages}{LaTeX, HTML, CSS, XML}
\cvitem{System administration}{Linux, cloud computing, virtual machines}
\cvitem{Software for qualitative analysis}{AtlasTI}
\cvitem{Office and graphics}{Prevalent office, DTP, and graphic software}


\section{Grants}
\cvitem{JEDS: Tracking the filter bubble}{I am co-applicant of the project ``Tracking the filter bubble'', funded with 244k Euro in cash and 2.5 FTE working time of research engineers in kind. Funded by the Dutch eScience Center. Resulted in hiring a PhD candidate, which I supervise (in 2018).}

\cvitem{Research time for lecturers} {0.2 FTE research time funded while working as lecturer, granted in a competitive round for lecturers with the best research proposal. Resulted in publication of article \cite{vanklingeren2017}. (granted per 9/2013)}

\cvitem{Competitive internal funding}{For the INCA project which I lead (``Infrastructure for Content Analysis'', \url{https://github.com/uvacw/inca}), we received in competitive funding rounds in total $24250$ Euro (2018: $2500$, 2017: $16750$, 2016: $5000$)}

%\cvitem{Award for best evaluated methods course}{UITZOEKEN WANNEER}
%\cvitem{Award for most helpful colleague}{UITZOEKEN WANNEER}


\section{Project leadership}
I am leader of the INCA project (``Infrastructure for Content Analysis'', \url{https://github.com/uvacw/inca}), in which we develop and maintain an infrastructure for automated content analysis, used by researchers across the department. I continuously supervise student assistants, interns, and scientific staff that contribute to the project. 


\section{Supervision}
\subsection{PhD supervision}
\cvitemwithcomment{Felicia L\"ocherbach}{``Tracking the Filter Bubble''}{2018-2022}%\newline
(co-supervision with Judith M\"oller and Wouter van Atteveldt) \newline

\cvitemwithcomment{Susan Vermeer}{``News for you!''} {2018-2021}
(co-supervision with Sanne Kruikemeier and Claes de Vreese) \newline

\cvitemwithcomment{Tom Dobber}{``Extent and Consequences of Political Behavioral Targeting''}{2016-2019}
(co-supervison with Natali Helberger and Claes de Vreese) \newline

\subsection{Bachelor and Master supervision}
I supervised 98 Bachelor and master theses (available at \url{http://scriptiesonline.uba.uva.nl/}). 

I additionally serve as regular second reader.



\section{Service to the discipline}
\cventry{2018--present}{Computational Communication Research}{Founding associate editor}{}{}{}

\cventry{2017--present}{Journalism, Media and Globalisation (Erasmus Mundus Master's joint degree)}{Member of Programme Committee (Opleidingscomissie)}{}{}{}

\cventry{2017--present}{Second reader to guarantee quality of master theses}{Member of Committee ``Thesis Quality Master''}{}{}{}

\cventry{2015--present}{Advising College and Gradudate School regarding the methods curriculum}{Member of Methods Committee}{}{}{}

\cventry{2016}{Developing curriculum for new Master program at the Faculty of Science}{Member of the curriculum committee MSc Data Science}{}{}{}

\cventry{2015--2016}{JongUvA -- committee to organize social events for young colleagues}{Member of Social Commitee}{}{}{}

\cventry{continously}{Regular reviewer of manuscripts for conferences and journals, including:}{Ad-hoc reviewer}{}{}{}
Journal of Communication, Social Science Computer
Review, New Media \& Society, Mass Communication and Society,
Communication Methods and Measures, Medien \&
Kommunikationswissenschaft, Tijdschrift voor
Communicatiewetenschap, International Journal of Public Opinion
Research, International Journal of Communication,
Communications: The European Journal of Communication
Research, Journal of Broadcasting and Electronic Media, Mobile
Media \& Communication, Journal of Computer-Mediated
Communication, First Monday;
ICA, ECREA, WAPOR, Etmaal van de Communicatiewetenschap.

\emph{Also reviewing for several funding agencies, including the Czech
Science Foundation and the Flemish Science Foundation.} \newline





\section{Teaching}
%\cvitemwithcomment{bla}{natirrerjerje}{Comment}
\cvitem{Big Data and Automated Content Analysis}{Research Master methods course. Taught every academic year since 2013/2014, in semester 2.}

\cvitem{Innovating Journalism}{Master course on data journalism. Taught with Penny Sheets in academic year 2018/19, semester 1.}

\cvitem{Journalistic Product}{Practical course to turn Master's thesis into a journalistic product. Taught in academic year 2016/17, semester 2, and academic year 2017/18, semester 2.}

\cvitem{Media Ethics}{MOOC (Massive Open Online Course). Developed with Rutger de Graaf, Elgin Blankwater, Annemarie van Oosten, Sandra Jacobs, Lotte Salome. Continously available online.}

\cvitem{Introduction to Research Methods}{Seminar group in the pre-master program. Taught in academic year 2014/15, semester 1.}

\cvitem{Hot topics in political communication}{Bachelor graduation project (Afstudeerproject). Taught with Linda Bos in academic year 2014/15, semester 1.}

\cvitem{Graduation project Excellence track (Afstudeerproject Excellentietraject)}{Bachelor graduation project for excellent students. Taught in academic year 2013/14, semester 2 and academic year 2014/15, semester 2.}

\cvitem{Citizens and Public Opinion}{Master specialization seminar. Taught with Anouk van Drunen and Regula H\"anglli in academic year 2012/13, semester 2.}

\cvitem{Journalism and the Media}{Master specialization seminar. Taught with Richard van der Wurff in academic year 2012/13, semester 1.}

\cvitem{New Media, New Politics?}{Master elective. Taught in academic year 2012/13, semester 1 and academic year 2012/13, semester 2.}

\cvitem{Domain Module Political Communication and Journalism}{Bachelor seminar groups (Werkgroepen Domeinmodule politieke communicatie en journalistiek). Two groups taught in academic year 2012/13, semester 1; two groups taught in academic year 2015/16, semester 1.}

\cvitem{Dead Trees and Digital Citizens: News and Journalism in an Online Era}{Bachelor graduation seminar (Afstudeerseminar Dode bomen en digitale burgers: Nieuws en journalistiek in het online tijdperk). Taught with Tom Bakker in academic year 2011/12, semester 2; taught as graduation project (afstudeerproject) in academic year 2012/13, semester 2 and academic year 2013/14, semester 1.}

\cvitem{Introduction to Communication Science} {Bachelor seminar group (Werkgroep Inleiding Communicatiewetenschap). Taught in academic year 2010/11, semester 1.}


\section{Invited talks and workshops (selection)}

\cventry{9-2017}{Four-day workshop Automated Content Analysis with Python}{Universiteit Antwerpen}{}{}{}

\cventry{14-9-2017}{Invited speaker at Workshop Medien in Konflikten.}{Schader-Forum, Darmstadt.}{}{}{}

\cventry{5-2017}{Two-day workshop Automated Content Analysis with Python}{Radboud Universiteit, Nijmegen}{}{}{}

\cventry{10-4-2017}{Invited presentation:``Filter bubbles are overrated''}{Vrije Universiteit, Amsterdam.}{}{}{}

\cventry{27-3-2017}{Participant in a panel discussion on filter bubbles and algorithmic news selection}{Pakhuis de Zwijger, ``Sign of Time'' series, Amsterdam}{}{}{}

\cventry{2/2017}{Two-day workshop Automated Content Analysis with Python}{Amsterdam Institute for Social Science Research, Universiteit van Amsterdam}{}{}{}

\cventry{12-5-2016}{Session at PhD training workshop: ''Data analysis in the digital age''
%”Qualitative research methods in the digital era: How
%interviews have been re-shaped by technology?
}{Geneva,
Switzerland}{}{}{} 
 
\cventry{4-2016}{Invited participant at expert workshop ``Analysis, Interpretation and Benefit of  \emph{User-Generated Data: Computer Science Meets Communication Studies''}}{Schloss Dagstuhl, Germany}{}{}{}

\cventry{1-3-2016}{Invited talk: ``Inductive automated frame analysis''
%.  at “Expert meeting on automatic text analysis”,
}{Utrecht School of Governance, Utrecht
University}{}{}{}
 
 \cventry{24-2-2016}{Invited talk: ``Setting up an infrastructure for large-scale automated content analysis''}{Computational Social Science Meetup, Freie Universität Berlin}{}{}{}

\cventry{8-12-2015}{Guest lecture: ``Big Data \& network analysis'' %department of Political Science at RU Nijmegen
}{Radboud Universiteit, Nijmegen}{}{}{}

\cventry{1-12-2015}{Invited talk at pre-symposium workshop: \emph{``From word frequencies to topic modeling: Applying automated content analysis}  \emph{techniques to short social media messages''}  }{GESIS Computational Social Science Winter Symposium, Cologne}{}{}{}

\cventry{27-11-2015}{Invited lecture: ``Packing and unpacking the bag of words: Introducing a toolkit for inductive automated frame analysis.''}{IdeaLabsSymposium, ``Social media: incubators of a renewed news media landscape?'', KU Leuven, Belgium}{}{}{}

\cventry{22-9-2015}{Invited talk: ``Meer dan sentimentscores: Inzichten destilleren uit een enorme hoeveelheid data''}{Nederlandstalig Platform voor Surveyonderzoek (NPSO), Amsterdam}{}{}{}

\cventry{8-10-2014 }{Workshop ``Python in the Social Sciences''}{Utrecht Data School, Utrecht University}{}{}{}

\cventry{5-3-2014}{Workshop ``Python in the Social Sciences''}{Coding Culture, Utrecht}{}{}{}

\cventry{3-2014}{Hands-on-Workshop Analyzing Big (Twitter) Data.}{Department of Communication Science, Universiteit van Amsterdam}{}{}{}
 


%\section{Extra 1}
%\cvlistitem{Item 1}
%\cvlistitem{Item 2}
%\cvlistitem{Item 3. This item is particularly long and therefore normally spans over several lines. Did you notice the indentation when the line wraps?}

%\section{Extra 2}
%\cvlistdoubleitem{Item 1}{Item 4}
%\cvlistdoubleitem{Item 2}{Item 5\cite{book1}}
%\cvlistdoubleitem{Item 3}{Item 6. Like item 3 in the single column list before, this item is particularly long to wrap over several lines.}

%\section{References}
%\begin{cvcolumns}
%  \cvcolumn{Category 1}{\begin{itemize}\item Person 1\item Person 2\item Person 3\end{itemize}}
%  \cvcolumn{Category 2}{Amongst others:\begin{itemize}\item Person 1, and\item Person 2\end{itemize}(more upon request)}
%  \cvcolumn[0.5]{All the rest \& some more}{\textit{That} person, and %\textbf{those} also (all available upon request).}
% \end{cvcolumns}


\section{Publications}



\subsection{Manuscripts under review}

\makeatletter
\long\def\thebibliography#1{%
  %\section*{\refname}%
  %\@mkboth{\MakeUppercase\refname}{\MakeUppercase\refname}
  \settowidth{\dimen0}{\@biblabel{#1}+5}%  ADD OFFSET (5 in dit geval, gewoon gegokt) zodat het nog steeds goed is uitgelijnd
  \setlength{\dimen2}{\dimen0}%
  \addtolength{\dimen2}{\labelsep}
  \sloppy
  \clubpenalty 4000 
  \@clubpenalty 
  \clubpenalty 
  \widowpenalty 4000
  \sfcode `\.\@m
  \renewcommand{\labelenumi}{\@biblabel{R \theenumi}} % labels like [3], [2], [1]
  \begin{etaremune}[labelwidth=\dimen0,leftmargin=\dimen2]\@openbib@code
}
\def\endthebibliography{\end{etaremune}}
\def\@bibitem#1{%
  \item \if@filesw\immediate\write\@auxout{\string\bibcite{#1}{\the\value{enumi}}}\fi\ignorespaces
}
\makeatother


\bibliographystylereview{unsrt}  
\bibliographyreview{review}                  






\subsection{Peer-reviewed articles}

\makeatletter
\long\def\thebibliography#1{%
  %\section*{\refname}%
  %\@mkboth{\MakeUppercase\refname}{\MakeUppercase\refname}
  \settowidth{\dimen0}{\@biblabel{#1}+5}%  ADD OFFSET (5 in dit geval, gewoon gegokt) zodat het nog steeds goed is uitgelijnd
  \setlength{\dimen2}{\dimen0}%
  \addtolength{\dimen2}{\labelsep}
  \sloppy
  \clubpenalty 4000 
  \@clubpenalty 
  \clubpenalty 
  \widowpenalty 4000
  \sfcode `\.\@m
  \renewcommand{\labelenumi}{\@biblabel{A \theenumi}} % labels like [3], [2], [1]
  \begin{etaremune}[labelwidth=\dimen0,leftmargin=\dimen2]\@openbib@code
}
\def\endthebibliography{\end{etaremune}}
\def\@bibitem#1{%
  \item \if@filesw\immediate\write\@auxout{\string\bibcite{#1}{\the\value{enumi}}}\fi\ignorespaces
}
\makeatother





%\nocitebook{Trilling2016c}
%\bibliographystylebook{plain}
%\bibliographybook{publications}                   % 'publications' is the name of a BibTeX file
%\nocitemisc{misc1,misc2,misc3}
%\bibliographystylemisc{plain}
%\bibliographymisc{publications}                   % 'publications' is the name of a BibTeX file

\bibliographystyleart{unsrt}  
\bibliographyart{articles}   




\subsection{Theses}


\makeatletter
\long\def\thebibliography#1{%
  %\section*{\refname}%
  %\@mkboth{\MakeUppercase\refname}{\MakeUppercase\refname}
  \settowidth{\dimen0}{\@biblabel{#1}+5}%  ADD OFFSET (5 in dit geval, gewoon gegokt) zodat het nog steeds goed is uitgelijnd
  \setlength{\dimen2}{\dimen0}%
  \addtolength{\dimen2}{\labelsep}
  \sloppy
  \clubpenalty 4000 
  \@clubpenalty 
  \clubpenalty 
  \widowpenalty 4000
  \sfcode `\.\@m
  \renewcommand{\labelenumi}{\@biblabel{T \theenumi}} % labels like [3], [2], [1]
  \begin{etaremune}[labelwidth=\dimen0,leftmargin=\dimen2]\@openbib@code
}
\def\endthebibliography{\end{etaremune}}
\def\@bibitem#1{%
  \item \if@filesw\immediate\write\@auxout{\string\bibcite{#1}{\the\value{enumi}}}\fi\ignorespaces
}
\makeatother



\bibliographystylethes{unsrt}    % of unsrt
\bibliographythes{theses}                  


\subsection{Textbooks}

\makeatletter
\long\def\thebibliography#1{%
  %\section*{\refname}%
  %\@mkboth{\MakeUppercase\refname}{\MakeUppercase\refname}
  \settowidth{\dimen0}{\@biblabel{#1}+5}%  ADD OFFSET (5 in dit geval, gewoon gegokt) zodat het nog steeds goed is uitgelijnd
  \setlength{\dimen2}{\dimen0}%
  \addtolength{\dimen2}{\labelsep}
  \sloppy
  \clubpenalty 4000 
  \@clubpenalty 
  \clubpenalty 
  \widowpenalty 4000
  \sfcode `\.\@m
  \renewcommand{\labelenumi}{\@biblabel{TB \theenumi}} % labels like [3], [2], [1]
  \begin{etaremune}[labelwidth=\dimen0,leftmargin=\dimen2]\@openbib@code
}
\def\endthebibliography{\end{etaremune}}
\def\@bibitem#1{%
  \item \if@filesw\immediate\write\@auxout{\string\bibcite{#1}{\the\value{enumi}}}\fi\ignorespaces
}
\makeatother



\bibliographystyletextbook{unsrt}    % of unsrt
\bibliographytextbook{textbooks}                  





\subsection{Book chapters}


\makeatletter
\long\def\thebibliography#1{%
  %\section*{\refname}%
  %\@mkboth{\MakeUppercase\refname}{\MakeUppercase\refname}
  \settowidth{\dimen0}{\@biblabel{#1}+5}%  ADD OFFSET (5 in dit geval, gewoon gegokt) zodat het nog steeds goed is uitgelijnd
  \setlength{\dimen2}{\dimen0}%
  \addtolength{\dimen2}{\labelsep}
  \sloppy
  \clubpenalty 4000 
  \@clubpenalty 
  \clubpenalty 
  \widowpenalty 4000
  \sfcode `\.\@m
  \renewcommand{\labelenumi}{\@biblabel{C \theenumi}} % labels like [3], [2], [1]
  \begin{etaremune}[labelwidth=\dimen0,leftmargin=\dimen2]\@openbib@code
}
\def\endthebibliography{\end{etaremune}}
\def\@bibitem#1{%
  \item \if@filesw\immediate\write\@auxout{\string\bibcite{#1}{\the\value{enumi}}}\fi\ignorespaces
}
\makeatother


\bibliographystylechapter{unsrt}   %ofunsrtdin dan krijg je o.a. ook doi 
\bibliographychapter{bookchapters}                  





\subsection{Other publications}

\makeatletter
\long\def\thebibliography#1{%
  %\section*{\refname}%
  %\@mkboth{\MakeUppercase\refname}{\MakeUppercase\refname}
  \settowidth{\dimen0}{\@biblabel{#1}+5}%  ADD OFFSET (5 in dit geval, gewoon gegokt) zodat het nog steeds goed is uitgelijnd
  \setlength{\dimen2}{\dimen0}%
  \addtolength{\dimen2}{\labelsep}
  \sloppy
  \clubpenalty 4000 
  \@clubpenalty 
  \clubpenalty 
  \widowpenalty 4000
  \sfcode `\.\@m
  \renewcommand{\labelenumi}{\@biblabel{O \theenumi}} % labels like [3], [2], [1]
  \begin{etaremune}[labelwidth=\dimen0,leftmargin=\dimen2]\@openbib@code
}
\def\endthebibliography{\end{etaremune}}
\def\@bibitem#1{%
  \item \if@filesw\immediate\write\@auxout{\string\bibcite{#1}{\the\value{enumi}}}\fi\ignorespaces
}
\makeatother


\bibliographystylemisc{unsrt}  
\bibliographymisc{misc}                  







\subsection{Conference presentations and posters}


\makeatletter
\long\def\thebibliography#1{%
  %\section*{\refname}%
  %\@mkboth{\MakeUppercase\refname}{\MakeUppercase\refname}
  \settowidth{\dimen0}{\@biblabel{#1}+5}%  ADD OFFSET (5 in dit geval, gewoon gegokt) zodat het nog steeds goed is uitgelijnd
  \setlength{\dimen2}{\dimen0}%
  \addtolength{\dimen2}{\labelsep}
  \sloppy
  \clubpenalty 4000 
  \@clubpenalty 
  \clubpenalty 
  \widowpenalty 4000
  \sfcode `\.\@m
  \renewcommand{\labelenumi}{\@biblabel{P \theenumi}} % labels like [3], [2], [1]
  \begin{etaremune}[labelwidth=\dimen0,leftmargin=\dimen2]\@openbib@code
}
\def\endthebibliography{\end{etaremune}}
\def\@bibitem#1{%
  \item \if@filesw\immediate\write\@auxout{\string\bibcite{#1}{\the\value{enumi}}}\fi\ignorespaces
}
\makeatother


\bibliographystyleconf{unsrt}  
\bibliographyconf{conferences}                  






\clearpage








\end{document}
